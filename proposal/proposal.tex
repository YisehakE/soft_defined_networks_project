\documentclass{hotnets23}

\usepackage{times}  
\usepackage{hyperref}
\usepackage{titlesec}

\hypersetup{pdfstartview=FitH,pdfpagelayout=SinglePage}

\setlength\paperheight {11in}
\setlength\paperwidth {8.5in}
\setlength{\textwidth}{7in}
\setlength{\textheight}{9.25in}
\setlength{\oddsidemargin}{-.25in}
\setlength{\evensidemargin}{-.25in}

\begin{document}
%\begin{sloppypar}

% \conferenceinfo{HotNets 2023} {}
% \CopyrightYear{2023}
% \crdata{X}
% \date{}

%%%%%%%%%%%% THIS IS WHERE WE PUT IN THE TITLE AND AUTHORS %%%%%%%%%%%%

\title{Project proposal: SDN Telemetry \& Analysis}

\author{Ameya Dehade, Yisehak Ebrahim}

\maketitle

\section*{Problem statement}

\begin{itemize}
\item {\textbf{What is the exact problem you are trying to solve?} 
Networks can display unexpected behaviours, under normal \& abnormal conditions, which are useful for administrators to identify. Using a P4-based approach, network telemetry can help us address efficient distribution of network traffic in data center environments. Data center networks often face congestion issues due to high volume of traffic generated by various applications \& services hosted within them. Congestion not only degrades network performance but also impacts the reliability \& responsiveness of critical services. Despite efforts to increase link capacities \& optimize routing algorithms, persisting congestion highlights the need for innovative solutions. \cite{WhatisTelemetry}}

\item {\textbf{Why is the problem important?} The software-defined networking (SDN) network needs to carry more and more services as its scale increases, and users have placed higher requirements on intelligent SDN O\&M. Specifically, data monitoring requires a sampling with a higher precision, so that microburst traffic can be efficiently detected and adjusted. Additionally, the monitoring process should have little impact on device functions and performance in order to improve device and network utilization.\cite{WhatisTelemetry}}

\item {\textbf{Why is the problem hard to solve?} Addressing performance issues in data center topologies poses several challenges, making it difficult to solve. Traditional approaches such as increasing link capacities or optimizing routing protocols have limitations in effectively mitigating congestion. Additionally, the complexity of modern data center architectures, with diverse workloads \& traffic patterns, exacerbates the challenge of designing scalable \& adaptive solutions. Thus, our problem requires innovative \& adaptive strategies that can dynamically monitor changing network conditions \& traffic demands..}

\end{itemize}

\section*{Context}
This project aims to develop a P4-based network telemetry software to collect real-time performance data from network devices and evaluate its effectiveness in providing actionable insights into network health and behavior.

\textbf{High-Level Approach:} 

\underline{Leveraging P4 and Mininet}: We will utilize the P4 programming language to embed telemetry functionalities directly within the data plane of network devices. This in-band network telemetry (INT) approach allows for efficient and fine-grained collection of performance metrics like packet loss, latency, and throughput.\cite{InBand} We would also use Mininet to construct the different topologies and simulate different network conditions that we need, and use Pox as the OpenFlow controller.

\underline{SDN Integration:} While SDN is not strictly mandatory for this project, it can play a complementary role.  We can leverage an SDN controller to manage and configure P4 switches, enabling dynamic adaptation of the telemetry collection based on network conditions.

\underline{Data Collection \& Analysis:} The P4 program will generate telemetry data packets containing the collected metrics. We will develop a dedicated collector application to receive and process these data packets. This collector will likely be written in Python and utilize libraries for network communication and data analysis.

\underline{Actionable Insights:} The processed telemetry data will be visualized and analyzed to identify network performance trends, bottlenecks, and potential issues. This information will be presented in a user-friendly format (like graphs and charts), allowing network operators to take corrective actions and optimize network performance.


\section*{Related work}

In-band telemetry has been utilized before using an ONOS controller \cite{ONOS}. We'll try to build upon this work using the POX controller \& create a user-friendly interface to analyze the network statistics.

\bibliographystyle{abbrv} 
\begin{small}
\bibliography{ref}
\end{small}

%\end{sloppypar}

\end{document}

